\documentclass[12pt]{report}
\usepackage{url}
\usepackage{hyperref}
\usepackage{graphicx}
\usepackage[top=1.0in, bottom=1.0in, left=1.5in, right=1.5in]{geometry}
\title{Investigating the correlation between the success of video games and user interface design principles}
\author{Sean McLellan\\ SE4HC3\\ McMaster University}
\date{\today}
\parskip=8pt
\begin{document}
\maketitle

\chapter{Introduction}
\section{Abstract}
When the field of user interface design study was in its emergence, much of the technology available was also in its infancy. Over the course of decades, the study of controllable user interfaces has grown to encompass many components of the functional aspects of a system as well as the purely aesthetic elements of user interface design. As the study of interfaces continues to grow, an important correlation should be made regarding the increasing popularity of video games. Through the study of video games, and how the mass market critically accepts or rejects specific video game products, valuable information regarding what determines successful user interface design and immersion is available in vast quantities simply due to the enormous size of the video game industry. The video game industry took in approximately \$17 billion worth of sales in 2011\cite{GeorgSzalai}, with an industry value of approximately \$74 billion the same year, expected to grow to \$115 billion over the next 4 years\cite{DavidHinkle}. Additionally companies in the video game industry have to force themselves to constantly stay relevant to the current culture and user needs due to quickly evolving technology as well as the competition. The video game industry is so fickle that only the top 5\% of games published actually make a profit\cite{PaulGillin}. Through careful analysis of mainstream video games, designers can gain a greater understanding of how users perceive a new piece of software and how to provide the user with an experience that is quickly and easily adopted.

\section{Introduction}
There exists a general set of principles, derived through the study of user interfaces, identifying what a designer can do to create an efficient user tool that not only provides functionality but attempts to motivate the user, beneficial to their cognitive functionality. Shneiderman defines a system that employs direct manipulation as one that\cite{Shneiderman}:
\begin{enumerate}
\item Continuously represents the objects and actions of interest
\item Physical actions or presses of labelled buttons are used rather than using complex syntax
\item Immediately visible incremental reversible operations can be used rapidly on an object of interest
\end{enumerate}
Based on the definition provided, a video game is a form of direct manipulation. Other products that can be considered a form of direct manipulation include virtual reality (VR), augmented reality, and most computer user interfaces (word processors, video editing, etc.). Since these various types of user interfaces are branded under the same definition, what is to prevent principles derived specifically for one type of interface from being applied to another under the same definition? As an example, the study of virtual reality has grown considerably with the intervention of the video game industry. The effect the video game industry has had on VR is so large that other VR researchers have had to redirect their focus towards game research and development in order to remain relevant, embracing the ideas gained from observing the video game community and applying that to all matter of scenarios such as government and corporate training and simulation\cite{Zyda}. The impact that video games have had on the world is remarkable but to truly understand why video games are able to stimulate other fields, we must look at what successful video games have done to their user design that other fields have not.

In the video game industry, the measures of usability are given different prioritizations. Rather than judging the usability of a design on efficiency, effectiveness and satisfaction equally, satisfaction of the user is given precedence among all other elements\cite{Federoff}. Video games are separated from the other types of software available to the world in the sense that video games do not serve a real-world purpose other than to entertain\cite{Federoff}. By observing this trait, it can be assumed that one of the most important elements of design for an interface is its ability to immerse the user in the world, and failure to do so can result in a failure of a product. By examining video games throughout history using a set of heuristics, it should be possible to find exactly what elements of design are most important and how much of an impact good interface design can have on the success of a game.

\chapter{Methodology}
\section{Methodology}
To gain a greater understanding of the effects user interface design can have on a product, this process will dissect a relevant and mainstream game in order to determine the correlation between interface, user satisfaction, video game success, and to compare the effect user interface design has on success versus other elements, specifically game play. It is possible that the impact user interfaces have on overall video game success is minimal, and failure may be attributed to many other factors such as game story and game play. For the purpose of this investigation, we will classify a video game interface as an element which encompasses the physical control of a game (such as a mouse and keyboard or a joystick), as well as the visual representation of control actions and user information\cite{Federoff}.

The set of heuristics this analysis will use to compare the successfulness of a video game interface and play implementation is a compilation of rules created by other researchers either in the field of user interaction research or the video game industry.

There will be one video game analysed and referenced in this comparison: Star Wars The Old Republic (Bioware). The comparisons will only take into account the condition the game was in when it was \textit{released} not the condition the game is in at its current state. This video game was chosen because it is a relatively recent example, it is a massively multiplayer online (MMO) games meaning that it has/had a large player base, and it had a shift in financial strategy as well as player feedback versus reviewer feedback. Finally, the industry standard will be World of Warcraft, developed by Blizzard/Activision, because it is the largest and most successful MMO game to date, with an estimated 10 million subscribers in 2012\cite{Karmali}.

\section{Heuristics}

With the tables of heuristics provided, evaluating the success of the game in question will be a combination of comparing the game's release state with the list of heuristics, similar to that of a check-list, and when elements of the game do not meet the criteria set out by this table that element will be discussed. Elements that a game does particularly well will also be discussed and after all heuristics are evaluated, the success that the game received will be compared to the heuristics which were not met to see if a correlation can be drawn between video game success and compliance with the list of usability heuristics for the game interface or game play.

\begin{table}[ht]
\caption{Heuristics for Usability of a Game Interface} % title of Table
\centering  % used for centering table
\begin{tabular}{c p{\textwidth}}
\hline\hline                        %inserts double horizontal lines
No. & Description \\ [0.5ex] % inserts table 
%heading
\hline                  % inserts single horizontal line
1 & Provide immediate feedback for user actions.\cite{Desurvire}  \\ % inserting body of the table
2 & The player experiences the user interface as consistent (in control, colour, typography, and dialogue design) but the game play is varied.\cite{Desurvire}  \\
3 & The player should experience the menu as part of the game.\cite{Desurvire}  \\
4 & Players do not need to use a manual to play the game.\cite{Desurvire}  \\
5 & The interface should be as non-intrusive to the player as possible.\cite{Desurvire}  \\ 
6 & Art should be recognizable to the player, and speak to its function.\cite{Desurvire} \\
7 & Make the menu layers well-organized and minimalist to the extent the menu options are intuitive.\cite{Desurvire} \\
8 & Sounds from the game provide meaningful feedback or stir a particular emotion.\cite{Desurvire} \\
9 & Get the player involved quickly and easily with tutorials.\cite{Desurvire} \\
10 & Allow users to customize video and audio settings.\cite{Pinelle} \\ 
11 & Allow users to skip frequently repeated content.\cite {Pinelle} \\
12 & Controls should be customizable and default to industry standard settings. \cite{Federoff} \\ [1ex]% [1ex] adds vertical space
\hline %inserts single line
\end{tabular}
\label{tab:nonlin} % is used to refer this table in the text
\end{table}

\begin{table}[ht]
\caption{Heuristics for Evaluating Game Play} % title of Table
\centering  % used for centering table
\begin{tabular}{c p{\textwidth}}
\hline\hline                        %inserts double horizontal lines
No. & Description \\ [0.5ex] % inserts table 
%heading
\hline                  % inserts single horizontal line
13 & Player's fatigue is minimized by varying activities and pacing during game play.\cite{Desurvire}  \\ % inserting body of the table
14 & The game is enjoyable to replay.\cite{Desurvire}  \\
15 & Players discover the story as part of the game play.\cite{Desurvire}  \\
16 & Player's should perceive a sense of control and impact onto the game world.\cite{Desurvire}  \\
17 & Vary the difficulty level so that the player has greater challenge as they develop mastery.\cite{Desurvire}  \\ 
18 & The game should give hints, but not too many.\cite{Federoff} \\ [1ex]% [1ex] adds vertical space
\hline %inserts single line
\end{tabular}
\label{tab:nonlin2} % is used to refer this table in the text
\end{table}

\section{Case Study}
\subsection{Star Wars The Old Republic}
Star Wars The Old Republic failed to meet heuristic 12 which states that the controls should be customizable and default to the industry standard. The industry standard in this case would be referring to World of Warcraft, the largest MMO game to date, which allows the use of add-on tools to manipulate the arrangement of control schemes and UI elements such as the on screen buttons, minimaps, target bars, etc. Star Wars The Old Republic failed to launch with this criteria, and was subject to many complaints of this key feature missing from the game. It was not until patch 1.2, roughly 3-4 months after launching that the developers added the feature to the game. Other heuristics that have been arguably broken include number 5 on the list of interface heuristics, which stems from the fact that there is the ability to place 32 active skill buttons on the screen (4 action bars with 8 slots each). Although the use of all 4 action bars is optional, the amount of micromanagement required to activate all the abilities and to keep track of the cool-downs for each ability can be cumbersome while not actually adding any depth to the game itself. Although Star Wars The Old Republic did not manage to satisfy all conditions in table 2.1, it did handle many aspects of the interface heuristics well, the ones worth mentioning include heuristics 6 and 8. In Bioware's design process they specifically state during an interview that the goal with their character designs, animations, and sounds was to harmonize those parts of the game for specific skills to provide unique feedback to the player so that the player would not have to watch their action bars to see what skills were just used. Additionally, when the player is fighting against another player, the distinct skill animation and sound allows for players to easily identify what their enemy is using even in large group fights.

With regards to game play, Star Wars did not manage to satisfy heuristics 13, 14, 16, 17, and 18. The game is composed of a variation of \textit{fetch} quests requiring the user to go out into the game world and retrieve something or kill something and return to the quest giver. These types of quests are commonly used by RPG game developers however the general consensus is that players find these quests boring and needless, only serving the purpose of artificially lengthening the game\cite{FetchQuest}. This point applies to heuristic 13 and 14 as the game uses this quest archetype for the majority of missions available to the player. Additionally, the game has an overarching world story line that players must play through for each character they make. Initially this story is immerses the player however it has been a common complaint that much of the world content is boring to play through again causing many players to completely skip the dialogue involved. Another complaint from the game's message boards was that the players did not feel any impact on the actual game world since all quests and effects of those quests were isolated to the players involved even if the quest was to destroy a portion of a planet, the area destroyed would be back to normal directly after a cut-scene, which directly relates to heuristic 16. Other complaints from the message boards included the lack of difficulty (heuristic 17) that solo players encountered due to a lack of difficulty control on portions of the game, only allowing players who participated in special instanced quest lines called Flashpoints the ability to control the difficulty level. The final heuristic that was not satisfied was
that the game should not give too many hints. This rule also ties in with heuristic 13 and 17, and was another major complaint from the users of their message board. Many complained that the simplicity of the fetch quests were a replacement for real content and made the game seem stale for many players very quickly after launch. The idea of a fetch quest almost forces a game to break heuristic 18 because a location where an important game object is located is clearly identified on a minimap and all the user needs to do is follow the point to achieve the goal. 

\subsection{Analysing Success}

\begin{figure}
\begin{center}
\leavevmode
\includegraphics[scale=0.5]{chart_1.png}
\end{center}
\caption{Subscribers vs. Time \cite{Subnumbers}}
\label{fig:subvtime}
\end{figure}

If we analyse the trends in figure 2.1 regarding subscribers per month over the course of the first 5 months the game was released for we can identify some points of interest. The first thing to note is the substantial amount of players subscribed in the month of February, approximately 2 months after the game was released. It approximately took players 75 hours of total game time to reach the maximum level in the game\cite{Corwin} (which coincides with approximate completion of the main storyline). Based on statistics directly from Bioware, average player game time per day is 4-6 hours\cite{FredDutton} which means that it takes an average player approximately 2 weeks to reach the end of the game, meaning many players who purchased the game on launch were able to complete the game by January. It is also cited that within a month of release Bioware had approximately 2 million subscribers and many subscribers had 3 month subscriptions \cite{FredDutton}. With this information we can see that roughly around the time that player subscriptions were ending in February, the number of subscribers begins to drop because players were not renewing. Some causes of this could be the fact that the game was simply not difficult enough as stated in the heuristics analysis, resulting in the game being completed too quickly leaving many players who relied directly on solo content without anything to do. Further inspection of figure 2.1 reveals that when patch 1.2 was released around April, the addition of a completely customizable user interface did not have any significant impact on subscriber numbers. Bioware made no major changes to resolve any other unsatisfied heuristic and the subscriber numbers continued to decline to just below 1 million subscribers in July, showing a net loss of approximately 700000 subscribers from February to July.


\chapter{Conclusion}
\section{Conclusion}
After completing a thorough comparison of the heuristics defined in the methodology, the game interface design, and the game play for Star Wars The Old Republic, a few key points can be interpolated based on the results. The actual effect that the design of video game user interfaces have on the success of a video game based on the results of the comparison is that if a large majority of the heuristics described in table 2.1 are followed, the satisfaction a player gets from the game goes up because more important elements of the game can be focused on by the player such as the actual visual stimulation of the environments and characters. However, when the interface heuristics are not followed, the player satisfaction with a game decreases as expected. If functionality is still present players will generally ignore the game interface's misstep. Based on the results of Star Wars The Old Republic, it is clear that when functional elements of a game do not meet the heuristics used to evaluate game play, the user base will quickly reject the game regardless of what changes to the interface are made. An argument can be made that the interface, game play and mechanics are not separate entities and should not be evaluated as such, instead the game should be considered the interface and vice-versa\cite{Federoff}. To conclude, although interface design is an important element of a video game and can result in dissatisfied players, the mechanics, functionality and play of a game trump the visual and auditory aspect of a game, making the effect interface has on the success of a game minimal and secondary to the other elements of game design.

\bibliographystyle{plain}
\bibliography{4HC3_BIB}

\end{document}