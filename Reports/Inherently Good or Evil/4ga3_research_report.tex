\documentclass[12pt]{report}
\usepackage{url}
\usepackage{hyperref}
\usepackage{graphicx}
\usepackage{comment}
\usepackage{enumerate}
\usepackage[top=1.0in, bottom=1.0in, left=1.5in, right=1.5in]{geometry}
\title{Are Humans Inherently Good or Evil: A Critical Analysis of Human Behaviour Through the Study of Virtual Simulations and Games}
\author{Sean McLellan\\ SE4GA3\\ McMaster University}
\date{\today}
\parskip=8pt
\begin{document}
\maketitle
\tableofcontents

\chapter{Introduction}
A Great English writer by the name of Oscar Wilde once said,
\begin{quotation}
Man is least himself when he talks in his own person. Give him a mask and he will tell you the truth.
\end{quotation}
This assertion can be built upon, opening the avenue for a more relevant question to be asked, "When man is himself, is his nature that of good or evil?"
This question has been the focus of many behavioural studies, however within this report the findings made will be based on observations through the use of a more modern medium of interaction and engagement. Video games are truly the first observable form of consumable media that allow the user to influence the output. Behavioural studies including non-game media were possible (i.e. music choice studies), however the conclusions drawn from pure consumption based studies are weak at best when compared to the results that can be drawn from directly observing the choices users make within the boundaries of a game, \textit{influencing} the very entertainment they are consuming. By assuming that the results drawn from a video game environment are valid with respect to behavioural studies, this report will dissect the outcome of various research experiments and eventually reach a conclusion regarding the primal nature of human cognitive processes.

\chapter{Methodology}
\section{Proposal}
For the purposes of this research report, the cognitive and social aspects of a player will be critically analysed. From the perspective of the player, the report will consider the benefits and consequences in each video game discussed and determine if those factors play a role in how the player ultimately behaves with respect to being morally good or evil. The report will also address the idea of primal instinct, and how players react when placed in a situation of duress. Additionally, the report will look into how players react when the response time and the time to process the situation is minimal. Within the realm of video games, there exists two general categories of play, \textit{player versus environment} (PVE) and \textit{player versus player} (PVP). This report will look at a variety of case studies observing how players behaved while playing these two types of games, and if the type of game affects the way a player makes decisions, ultimately affecting the moral judgement of a player. Also important to the cognitive state of a player are the types of communication channels available to players in each specific game (where player interaction is involved), and how they affect the judgement of a player. Finally, it is important to note whether or not the game does anything to exacerbate specific player behaviour, and whether or not this is intentional. The report also needs to make clear what constitutes good and evil, and how those categorizations are formed. The last element that determines whether or not using a video game/simulation can actually provide insight into how a human would behave outside of the virtual world has to do with perceived risks. If the game does not actually contain anything to lose, then can the player really be held accountable for their actions since their actions mean nothing. This type of risk can be the loss of a digital good, or real-world monetary credit. Through the analysis of all these factors, the study should be able to reach a conclusion as to whether humans are inherently good or evil.

\chapter{Discussion}

\section{Ethics, Morals and Empathy}
To begin discussing the results of the studies in this report, it is important that the basic components of the topic are made clear. Ethics are a set of personal or private standards that a being uses to guide what is acceptable and what is unacceptable in terms of how they behave and how people around them should behave.\cite{Ethical_Choices_in_Fable_3} Morality on the other hand is a set of guidelines universally defining what is right and wrong, however what is right and wrong comes down to the ethics of the individual.\cite{Wiki_Morality} Empathy is key to the way an individual interprets what is right and wrong categorically, specifically affecting the formation of an individual's ethical code. With regards to video games, player behaviour is directly affected by their ability to empathize with the environment and the other players that share the world they are inhabiting.

\section{Player Aggression}
Experimental studies have shown that violent video games have a tendency to increase aggressive behaviour in players, increase the hostile expectations a user has with regard to other players and has also shown to cause a decrease in helping behaviour within players.\cite{Exposure_to_violence_in_videogames} From a cognitive perspective, raised aggression results in behavioural changes and reactionary adjustments. Aggression has many factors which influence the degree/intensity a player feels at any given moment in a virtual environment. Since the level of aggression directly alters the player behaviour, it also indirectly affects the player's ethical code and ethical decision making abilities. 

The factors which affect player behaviour, that will also be discussed further, include:
\begin{enumerate}
\item Level of exposure to violence
\item Solo, Cooperative, or Competitive environments
\item Frustration
\item Rewards or Punishments
\item Communication Channels
\item Social Identity
\item Group Interaction
\item Gender
\end{enumerate}

This list of factors has been cultivated from various studies and where each factor has been identified as influential on player cognition. Each of these elements will be discussed in greater detail as well as the amount of influence each factor ultimately has with respect to player behaviour.

\subsection{Study: Automatic Aggressiveness from DOOM and Mah-jongg\cite{Exposure_to_violence_in_videogames}}

This study was conducted with a test group of one hundred and twenty-one introductory psychology students around the age of 18. Each participant played 10-minute equal intervals of the violent video-game \textit{Doom} and a computerized version of the puzzle game, \textit{Mah-jongg: Clicks}. After each 10-minute block, the participants in the study were asked to complete an implicit association test (IAT). The test required the participants to categorize a series of words under 4 headings: \textit{Self, Other, Aggressive and Peaceful}. One set of categorizations was to sort the words between self and other, and then separately sort between aggressive and peaceful. This was followed by one more categorization where participants had to sort the words into self-aggressive or other-peaceful. The more positive the score in the test, the stronger aggression and self were associated.

\begin{figure}[h]
\begin{center}
\leavevmode
\includegraphics[scale=0.5]{Automatic_Aggressive_Self_Concept.png}
\end{center}
\caption{Automatic Aggressive Self Concept by Gender and Game Condition \cite{Exposure_to_violence_in_videogames}}
\label{fig:exp_to_violence}
\end{figure}

Unsurprisingly, the results determined that playing violent video games were able to increase aggressiveness. However, the findings were relatively weak as the margin of error between the results for the non-violent game and violent game overlapped for the male participants. The weakness of the increase in aggressiveness may be attributed to a few things. The first and most obvious is the lack of play time. From the perspective of a gamer, ten minutes is barely enough to become acquainted with the controls and the motivation of the game, let alone for it to affect the cognitive functions of the participant. The other factor that may have played a role in the small margins gained before and after playing a violent game could be lack of immersion (which also relates back to time played). Since these games are \textit{solo play} games, the time required to fully immerse a player into the virtual world would generally take longer because interaction is up to the individual. An important factor to then analyse is the level of interaction between players in a multi-player environment, when initiating contact is not always a factor that a player can control. So specifically, the affect that presence of other players have on the aggression level of an individual.

\section{Cooperative and Competitive Play}
Many studies simply reduce game characteristics to the categorization of \textit{violent} and \textit{non-violent}, however aggression is affected by much more than just the exposure to violence a game presents to the player. Play mode has a large impact on the type of environment the game sets for the player, affecting the mood, goals, and ultimately the cognitive process of a player. If we examine the \textit{reward} structure in competitive modes, we can see that the way a game chooses to acknowledge success conditions the player to behave within a particular pattern (i.e. killing people in a game like \textit{Call of Duty} awards the player points).\cite{Killing_Spree} Another distinct connection between competitive games and aggression lies within competition's inherent ability to be the motivator/source of enjoyment for the player. Even when the player is not receiving enjoyment out of the game, but becoming rather frustrated instead, aggression levels are still raised. This is because frustration level is proportional to the aggressive level of a player as well, especially within the confines of a competitive environment; players impede the success of one another in order to raise their own level of success.\cite{Killing_Spree} In a study conducted by Eastin (2007), different player group sizes were tested to see how their aggression level changed when presented with different goals (i.e. fighting with or against the other members of the group). In terms of game play type, the results of the study demonstrated that aggression was higher when the conditions of the game created an environment of competition rather than cooperation.\cite{Killing_Spree} Another interesting finding in the study was that as the group sizes became larger, the groups became more aggressive. This may be attributed to the fact that as group sizes become larger, there is a greater chance of individuals having opposing ideas/goals, raising frustration levels which in turn raise the level of aggression.

\subsection{General Aggression Model}
The general aggression model is a structure designed to explain the effects that factors of aggression have on the cognitive stages of an individual. The general aggression model is an advancement over other models of aggression because this one factors in the idea that \textit{learning} may be a mechanism for the creation of aggression.\cite{Killing_Spree} The addition of learning as a factor allows the model to compensate for behaviours that individuals mimic from other players/agents within the virtual world, as well as behaviours that are learned when players are rewarded or punished for their actions. These punishments/rewards can even be self-administered, as an example if we look at the game \textit{WarZ} player behaviour can be attributed to the learning mechanisms. Players may initially want to be helpful and work together with other players, but because the game creates an environment where players cannot easily communicate with one another. This creates a shoot-first-ask-questions-later mentality. After the player is killed for no reason by another anxious player, they will learn that being helpful does not benefit them and its more rewarding to be violent.

\subsection{Study: Aggression and Play Mode in Halo: Combat Evolved\cite{Killing_Spree}}
This study took a test group and made each participant play \textit{Halo: Combat Evolved} under 3 different modes (\textit{Competitive, Cooperative, and Solo}). The participants played each mode for 30 minutes, and at the end of each segment they filled out a questionnaire. The questionnaire consisted of word completion, they were given the first two letters of each word and it was left up to the individual to complete the word (KI, DE, BL, ST, RI, SL). The answers to the questionnaire were evaluated by the testers, where aggressive cognition level was measured based on the amount of words associated with violence or aggression. The level of anxiousness and anger were also monitored during the experiment.

\begin{figure}[h]
\begin{center}
\leavevmode
\includegraphics[scale=0.5]{agression_in_halo.png}
\end{center}
\caption{Differences in Aggression, Frustration, and Violent Strategizing by Game Mode and Gender \cite{Killing_Spree}}
\label{fig:agg_in_halo}
\end{figure}

The results of the experiment demonstrated that competitive players showed the highest level of cognitive aggression regardless of gender, measurable data which supports the findings of Eastin's study. This can probably be attributed to the fact that competition against human agents requires more focus because humans generally behave unpredictably when compared to AI enemies. Solo players were found to be the most angry and frustrated when playing Halo. The reason for this has to do with the game mode itself. Although cooperative and solo play share a similar format (players work towards the end of the campaign), dying in solo play results in the player having to restart progress from the last checkpoint, where as dying in cooperative play allows the dead player to respawn near the other player as long as one of the members of the team remain alive. The fact that the solo player has a finite number of lives can result in a higher level of frustration and caution if the player dies too often, which would explain the results of the experiment. Based on the results of the study, the main attributer to increased aggression levels and factors of aggression is the instinct of survival in both competitive and solo play.

\section{Social Interaction and Hostile Expectations}
Game environments can exacerbate the issue of players becoming more aggressive and more violent. Sometimes these environments place the player in a state of greater hostility with the intent of raising their aggression.\cite{Hostile_expectations} By situating the player in an environment where the chance to face an element that will impede their progress exists, the player assumes hostile intentions of other agents in the virtual world, raising their level of aggressiveness by presuming the aggressiveness of others. An example of this environment is again in the game \textit{WarZ}. The channels of communication are bare and in an environment of violence the players can not be sure of the intentions of the other inhabitants of the game world, this causes the players to adopt hostile expectations, resulting in many aggressive behaviours such as killing rather than helping. Research on hostile expectations allows the general aggression model to be expanded to include group dynamics, another important factor in determining the state of cognition.

\begin{figure}[h]
\begin{center}
\leavevmode
\includegraphics[scale=0.5]{single_gam.png}
\end{center}
\caption{The single episode aggression model \cite{Hostile_expectations}}
\label{fig:single_eps_aggression_model}
\end{figure}

The single episode model of the general aggression model shows how the inputs of a situation (i.e. the game content) and the person (i.e. their gender, personality traits) influence the internal state of cognition and the decisions the player makes for each specific social situation.\cite{Hostile_expectations}

Williams and Skoric suggest that violent gameplay or exposure to violent gameplay in social multiplayer games such as \textit{Asheron's Call 2} has no real long-term affect on the player.\cite{Hostile_expectations} By citing Asheron's Call 2, they provide the example that players can not attack one another except in special zones located in the game world and would in fact frequently take the time to \textit{help} out others with less power or knowledge.
As a counter-example, by analysing the game environment of WarZ, when players are presented a world where the only \textit{designated zones} are for safety as opposed to combat, players generally assume a violent tendency towards other players when given the ability to decide between aiding and attacking.

\subsection{Inputs for Aggressive Cognition}
There are 4 more inputs in the general aggression model that need to be discussed. Trait hostility is the cognitive structure that exaggerates indications of hostility, taking elements of the game world that may not normally be interpreted as aggressive towards the player and making them become more aggressive in the eyes of the individual. Game experience is the input that affects the cognitive response of the player. As a player spends more time playing violent video games, they build up a natural response to certain situations, forming an automatic response that is generally more aggressive than players who do not spend that much time playing violent video games. Group size affects the competitive level of an individual, and as previously stated larger groups have more competition raising the level of aggression.

\begin{figure}[h]
\begin{center}
\leavevmode
\includegraphics[scale=0.5]{hostile_imp.png}
\end{center}
\caption{Proposed path model based on the general models of aggression \cite{Hostile_expectations}}
\label{fig:hostile_expectation_bias}
\end{figure}

The final input into aggressive cognition that has not been addressed yet is the element of in-game verbal aggression. Aggressive communication directly connects to aggressive behaviour (insults, competence attacks, teasing, profanity, ridicule, threats). Research suggest that individuals who verbalize aggression would also think and behave more aggressively.\cite{Hostile_expectations} Bales and Borgatta suggest that greater anonymity results in greater tension and antagonism because there is no social standing within a virtual environment. Since players do not care what other players think about them due to the fact that other players do not know who they are, their actions/behaviours can be more aggressive than they would be in the real world. Data from various studies indicates that when state hostility is heightened (the cognitive state of the player) by verbal aggression, trait hostility, and game experience, the player's hostile expectation bias increases.

It may also be possible that when a player engages in discourse with group members (verbal aggression), aggressive cognition leads the player to identify group members as competition on some form. Since the goal of competitive play is impeding the other player's progress, and since competitive play increases aggressive cognition more than cooperative play, the player may choose to throw a cooperative game (i.e. League of Legends) to prevent their own team from completing their goal. However, this is purely speculative and anecdotal.

\section{Griefing and Grief Play}
To understand good and evil, especially within the realm of a virtual world, it is important to identify a sub-play of all games that is generally considered \textit{evil}. This form of sub-play is called \textit{grief play}. However, griefing and grief play are not the same thing. Griefing is considered a disruptive cultural activity (i.e. hacking, crashing economies/servers, stealing characters, etc.). These activities are done outside of the boundaries of the game world, using third party programs to affect the virtual world and its inhabitants. Grief play on the other hand, stays within the boundaries of the game world, disrupting other player's gaming experiences.\cite{Goon_culture}

\subsection{Study: Griefing in Second Life}
By examining the game world of \textit{Second Life}, it can become clearer to understand the subculture of griefing within a game. Second Life is not the same as most other games because technically there is no "goal" to the game. Second Life is considered more of a platform for player interaction rather than a game. That being said games can exist within the world of Second Life, but the world is not bound by the rules of a game with a goal, the players create the purpose. Within the confines of the game environment, there exists a subset of players that participate in grief play, collectively trying to bring down the \textit{system}. Griefing has the ability to bring users together with similar goals, much like the way any other group of players form a community within a social game, except that the goal is detrimental to other players. Organized griefing becomes a culture, rather than just an isolated pathology, grouping similar minded players together to cooperatively or even competitively grief the game world.\cite{Goon_culture} Groups such as the \textit{Patriotic Nigras} from the message board \textit{4chan} would crash the Second Life servers regularly, while other lesser groups may just use offensive builds and images to grief other inhabitants of the game world. Even though these two groups are both griefing the environment, they are actually competing with one another both on how they grief as well as showing a their power to other griefing groups through the knowledge they possess (i.e. scripting).

Within the realm of Second Life, there exists a list of forbidden behaviour (a moral code per se), which acts as the universal guide of moral behaviour for all inhabitants of the game. 

This list of forbidden behaviour includes:
\begin{enumerate}
\item Intolerance
\item Harassment
\item Assault
\item Disclosure
\item Indecency
\item Disturbing the Peace
\end{enumerate}

Even though this guideline exists, the ethical codes of each subculture differ, and may not be in line with the moral code the game forces upon its users. Griefers are their own subculture, and many of their actions do not follow Second Life's guidelines. However, the server administrators take action against griefers to protect the rest of the community (usually by issuing bans).\cite{Griefing} Even with these punishments for behaviour, some players still grief because ultimately what is "Good and Evil" is determined by each subculture's ethical code (subjective) which determines player behaviour. The game's moral code forces players to behave a certain way regardless of what the ethical perspective of the individual is, much like laws and policing affect the real world.

\subsection{Deindividuation Theory}

When individuals assimilate with a crowd, their consideration for the consequences of their actions are diminished. This effect is caused because of the anonymity granted to all players within a game world. There are no social cues and social standing is irrelevant as it does not necessarily affect the reality of the individual. This allows the individuals within the game to use the communication channels to speak freely, with nothing held back since there are no real world consequences. However, Reicher predicted that anonymity does not make an individual lose awareness of their identity but instead shifts their awareness away from their personal identity to their social identity (i.e. perceiving themselves as part of a group).\cite{Griefing} The goals of the group and the ethics that group utilize are inherited by the individual, changing the way the individual would behave when compared to how they would behave alone, using their own code of ethics.

\section{Ethical Choices in Solo Play}
Ethical thinking is the ability for an individual to assess, interpret and reflect on decisions. The ability to empathize with other players is also an important component to ethical thinking. Being able to comprehend the complexities of ethical questions is ultimately the goal of ethical thinking.\cite{Ethical_Choices_in_Fable_3} The problems with examining ethical choices within a virtual world include a lack of real-world risks and the inclusion of morality meters. A perceived risk does not truly exist in most solo play, especially when the ability to save the game at any point exists. Players can make any decision they want regardless of what the outcome may be since the choice can be made again by simply reloading. The problem with morality meters (similar to the karma system in Fallout 3 or the renegade/paragon system in Mass Effect) is that they do not necessarily encourage ethical reflection, but instead focus the player into selecting a strategy that will maximize the amount of "good" or "evil" that can be acquired.\cite{Ethical_Choices_in_Fable_3}

\subsection{Study: Ethical Choices in Fable 3\cite{Ethical_Choices_in_Fable_3}}
A research experiment was conducted where a test group would have to play \textit{Fable 3} for at least 9 hours and then they had to complete a journal documenting their encounters in the game at 3 specific points in the game. The purpose of this study was to see how each participant handled 3 types of ethical dilemmas within the environment of a game.

The three scenarios were taken at different time points in the game, they included:
\begin{enumerate}
\item \textit{Surrender a Friend} - the main character (prince) has to decide between sacrificing their best friend or sacrificing the protesting villagers, and if no choice was made, both are killed.

\item \textit{Walter} - the main character's mentor has become blind while adventuring through a cave in the middle of the desert and either needs to be dragged across the desert or left behind.

\item \textit{Build a Brothel} - the main character has to decide between repairing an orphanage at the cost of 50000 gold to the kingdom's treasury or building a brothel which will earn the treasury one million gold. The catch is that for every dollar the treasury saves, a citizen will be protected in the upcoming war.
\end{enumerate}

\begin{figure}[h]
\begin{center}
\leavevmode
\includegraphics[scale=0.75]{surrender_a_friend.png}
\end{center}
\caption{Frequencies of Decisions on "Surrender a Friend" by Condition \cite{Ethical_Choices_in_Fable_3}}
\label{fig:surrender_a_friend}
\end{figure}

Gender and avatar identification affected the decisions of male participants. The male participants were significantly less likely to choose to save the male best friend, Elliot, when playing as a female. This could probably be attributed to the fact that the males could not empathize with the situation at hand, making it difficult to imagine a potential relationship with another male. The results of the "Surrender a Friend" scenario are in line with the theory that when a player is forced to make immediate decisions, they are generally more likely to make the self-less decision (in this case saving the villagers) rather than the selfish decision (saving their friend). All participants who sacrificed villagers did acknowledge that if they had more time to build a relationship with their friend, they would have been less likely to sacrifice the friend.\cite{Ethical_Choices_in_Fable_3} This is reflected in the results of the second scenario, which resulted in the majority of participants choosing to save Walter. This could be due to the fact that enough time had passed within the frame of the game allowing the relationship between the player and the mentor to become established and becoming valuable to the player, making the player choose to protect/keep alive what is important to them (survival).

\chapter {Conclusion}
Based on the research studies and data provided, to answer the question \textit{Are humans inherently good or evil?}, the conclusion can be drawn that humans are neither. The component that drives humans to do good or evil is the instinct to \textit{survive}, the ethical categorization is simply a bi-product of the decision made. Cognitive aggression is born from the desire to succeed and survive. Belonging is an important subconscious desire that offers safety and promotes aggression within the individual members of the group, fighting for the success/survival of the group/culture. The actual categorization of good and evil between cultures depends completely on the culture being examined and the behaviours that allow the culture to survive being classified as good or acceptable while behaviours that are detrimental to the culture being classified as evil or unacceptable. Finally, examining multiplayer interaction in video games is far more significant to the analysis of human behaviour than solo play because it offers a closer connection to reality as well as the element of social interaction and unpredictable human intervention.

\bibliographystyle{plain}
\bibliography{4ga3_bib}

\end{document}